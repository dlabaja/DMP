\section{Business model a škálovatelnost}

Ještě pár let zpátky většinu modelu tvořilo premium, kterým si uživatel mohl vylepšit účet a získat několik výhod včetně většího místa na disku pro své cesty. To se ale ukázalo jako nedostatečné, a tak je dnes příjem tvořen prodejem itinerářů, za které si Worldee bere 25\% marži.

Aby mohly cesty s průvodcem vůbec fungovat, jsou všichni průvodci placení (např. 2000\,Kč na den) a vedení jako Travel Buddy. Těmi se může stát kdokoliv, kdo vyplní přihlašovací formulář a projde náborovým procesem. Podobné je to s expaty, jejichž odměna je již přímo zahrnuta v ceně služeb. Protože má Worldee licenci a pojištění proti úpadku, mohou portál průvodci použít pro legalizaci svého podnikání a zároveň si stále budovat svoji značku. Ačkoliv bude jejich škálovatelnost pro expanzi do zahraničí náročná, protože budou potřeba stovky lidí různých národností, žádná platforma, která by něco podobného nabízela, na globální úrovni zatím neexistuje. A přitom mají tito průvodci vysokou přidanou hodnotu, protože pomáhají zákazníkům překonat jazykovou bariéru a~tím otevírají Worldee novým cílovým skupinám.

Daleko lépe škálovatelnější jsou cesty na vlastní pěst. Sám Tomáš o nich dokonce mluví jako o \uv{svatém grálu}. Vše by mělo do budoucna fungovat automaticky a data o~hotelech, letenkách a autech brát od třetích stran. Všechny itineráře už Worldee nyní umí automaticky překládat do všech světových jazyků. Stačí, aby si zákazník vybral termín, a~pak se nechal provádět mobilní aplikací v kapse.
