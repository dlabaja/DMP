\section{O firmě}

Co je vlastně to Worldee? Na to jsem si sám dlouho pokládal otázku. Jak jsem již říkal, ve zkratce je to taková cestovka 2.0. Nejlépe to ale asi popíše Tomáš Zapletal, její zakladatel, o nějž se v této části práce budu opírat.\cite{WorldeeInfo}

\begin{displayquote}
    \uv{Členové týmu Worldee vytváří platformu, která umožňuje lidem jednoduše objevovat svět. Spojením autenticity a jednoduchosti může díky naší platformě kdokoliv objevit téměř jakékoliv místo na zemi a cestovat v tom pravém slova smyslu. Naše poslání je ukázat lidem opravdový svět a pomoci jim odemknout skutečný potenciál jejich osobnosti - skrze zážitky, technologii a dokonalou zákaznickou zkušenost.}
\end{displayquote}

A jak to vlastně funguje? Worldee je takový hybrid mezi sociální sítí a cestovní kanceláří, který se skvěle doplňuje. Itineráře na stránce tvoří sami uživatelé a zkušení cestovatelé, a pokud se vám nějaký líbí, můžete si ho koupit. Worldee vyřídí vše od letenek, ubytování, auta a dalších služeb. Zároveň můžete cestovat s průvodcem nebo na vlastní pěst, s čímž vám pomůže jejich mobilní aplikace.

Cíl je jediný – stejně jako si pod rezervací hotelů představí člověk Booking.com, tak pod itineráři by se vám mělo vybavit právě Worldee.
\\
\begin{figure}[!h]
    \centering
    \includegraphics[width=0.3\linewidth]{obrazky/worldee.png}
    \caption[Logo firmy Worldee]{Logo firmy Worldee.\cite{Worldee}}
\end{figure}