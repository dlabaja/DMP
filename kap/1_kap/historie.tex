\section{Historie}

Vše začalo, když zakladatel firmy Tomáš Zapletal, potřeboval někam nahrát své zážitky z~cest.

\begin{displayquote}
    \uv{Jiné sociální sítě k ukládání cestovatelských zážitků (celých itinerářů) nebyly uzpůsobené. Facebook, Instagram, ani jiná sociální síť. No a vzhledem k tomu, že jsem žádnou takovou, která by mi vyhovovala, nenašel, založil jsem ji sám.}
\end{displayquote}

První prototyp vznikl v roce 2019 a vyvíjela ho externí firma, které Tomáš zaplatil asi dva miliony korun. Po prvním týdnu dostal na stránku přes 1500 uživatelů, které přilákal publikovaný článek na portálu \href{https://jaknaletenky.cz/cesky-startup-worldee-miri-do-sveta-a-potrebuje-vasi-pomoc.html}{jaknaletenky.cz}. K tomu vydal i \href{https://youtu.be/wJCV-5x0aIk}{první video} na \href{https://www.youtube.com/@worldee8910}{YouTube kanál Worldee}.

Časem bohužel zjistil, že se bez investice Worldee neposune. Společně se spoluzakladatelem Tomášem Nakládalem jezdili po republice a pokoušeli se získat 400 tisíc euro, které nakonec získali od Ondřeje Průši, zakladatele firmy prodávající známé Prusa 3D tiskárny.

První tým jedenácti lidí se v kanceláři poprvé sešel 1. 7. 2020. Vývoj a akvizice nových uživatelů šly pomalu. Největším problémem se ukázala být monetizace.

\begin{displayquote}
    \uv{Už existovalo mnoho startupů, které měly podobně ušlechtilou myšlenku - pomáhat lidem objevovat svět, pomocí plánovače, nebo cestovatelského deníku. Všechny ale shořely ve chvíli, kdy chtěly nápad monetizovat. Budoucnost tedy byla jasná - nejtěžším úkolem pro nás bude postavit kolem tohoto nápadu udržitelný business.}
\end{displayquote}

Nápad udělat z Worldee sociální síť tak částečně zkrachoval. Dobře nedopadl ani pokus o monetizaci mobilní aplikace. Jako nejlepší řešení se nakonec ukázal prodej cest – pokud se vám nějaký itinerář líbí, můžete si ho koupit jako balíček a bez starostí se na tu samou cestu vydat také. Byl to úspěch a první zájezd se vyprodal do 48\,hodin.

A tím se pomalu dostávám do současnosti, kdy jsem nastoupil i já. S Worldee byl ve Skotsku už i YouTuber Kovy se svým přítelem Mírou\cite{WorldeeKovy} a během toho přišla další, tentokrát stomilionová investice.\cite{WorldeeInvestice} Přispěl do ní zakladatel portálu Bazoš.cz Radim Smička a již podruhé společnost Kaiperi Venture Capital. Pomalu se rozjíždí expanze do zahraničí a~tým se rozrostl na více než padesát lidí. A jak říká název jedné z Tomášem zaregistrovaných firem – \uv{The sky is no longer the limit (s.\,r.\,o.)}.\cite{Podnikani}
