\section{Konkurence}

Ačkoliv je Worldee v Česku poměrně unikát, v zahraničí už existuje několik firem, které jsou s podobným projektem částečně úspěšné.

\subsection*{TripLegend} 
Jedná se o lokální německou cestovní kancelář založenou před čtyřmi lety. Nabízí cesty s průvodcem a spolupracují s cestovními agenturami, nedávno se začali pokoušet i o self-guided cesty. Zatím fungují v režimu break-even s jednotkami cest měsíčně, navíc bez jakéhokoliv vývojového týmu. Nicméně jsou důkazem, že podobný produkt může levně fungovat i na vysoce konkurenčním trhu.

\subsection*{\href{https://www.weroad.com/}{WeRoad}}
Tato cestovka má už mnohem větší dosah než TripLegend. Ačkoliv dělají jen cesty s~průvodcem doma a na některých západních trzích, a v balíčku ani nenabízí letenky, mají miliardový obrat a dokázali, že produkt může uspět i na chudším trhu, jako je Itálie.

\subsection*{\href{https://www.flashpack.com/}{FlashPack}}
Tento britský projekt, který se po pandemii dostal díky investorovi z insolvence, funguje hlavně na anglicky mluvících trzích. Staví si na prémiových službách a cenách, ale je možné, že se části z nich po saturaci trhu vzdají a díky penězům z marží expandují i na levnější trhy. Nicméně nemají ani vývojový tým, ani self-guided itineráře.

\subsection*{\href{https://www.polarsteps.com/}{PolarSteps}}
Jako poslední z firem tu uvedu projekt z Nizozemska. Podobně jako Worldee nabízí cestovatelský deník, ze kterého si ale navíc můžete nechat udělat fotoknihu. Zatím to pro Worldee může znamenat jen inspiraci pro další monetizaci, ale pokud své itineráře začnou prodávat, mohla by to být vzhledem k jejich velikosti potencionální hrozba.