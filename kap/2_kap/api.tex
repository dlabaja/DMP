\section{API}

Frontend je nyní de facto za Vámi. Abyste ale mohli komunikovat se serverem, který se stará o přístup do databáze a posílání stránek uživatelům, potřebujete nějaký univerzální most, který Vás s ním spojí A právě tomu se říká API\cite{WhatIsAPI}.
\todo{rozepsat různé druhy API}

Používáte ho denně, a ani o tom možná nevíte. Pokud potřebuje stránka dynamická data, musí o to požádat server HTTP požadavkem. Server po jeho vyhodnocení zase použije nějakou jinou API, díky níž se dostane do databáze, a až dostane odpověď, stejným HTTP požadavkem odpoví uživateli. Jeho stránka tak teď může konečně zobrazit data.

Důvodem, proč se většina dat posílá skrze JSON a XML, je jejich jednoduchá extrakce. Například knihovna HTMX ale přijímá data pouze v HTML. To může mít několik nevýhod. Data se z HTML těžko dostávají, a pokud je budete v tomhle formátu ukládat v databázi, pak už je nemůžete použít ani na jiné stránce, ani třeba v mobilní aplikaci. To je velká výhoda právě JSON formátu. Je podobně velký, jako kdybyste posílali normální text, ale zároveň z něj všude dokážete udělat datový objekt a používat jeho obsah.