\section{CSS}

Styly Vám na stránce umožní upravit skoro všechno – měnit barvy, odsazení, pozice, velikost, \ldots. Stačí jeden takový soubor připojit do hlavičky stránky. V něm definujete, co konkrétně chcete změnit. Zde uvedu příklad na tlačítku.

\begin{table}[h!]
\centering
\begin{tabular}{|>{\ttfamily}m{4cm}|m{8cm}|}
\hline
\textbf{Selektor} & \textbf{Popis} \\ \hline
\texttt{button \{\}} & Změní všechny \texttt{<button/>} elementy na stránce \\ \hline
\texttt{.save-button \{\}} & Změní všechny elementy se třídou (\textit{class}) \texttt{save-button} \\ \hline
\texttt{\#top-button \{\}} & Změní jeden konkrétní element s identifikátorem (\textit{id}) \texttt{top-button} \\ \hline
\end{tabular}
\caption{Příklady CSS selektorů a jejich funkcí.}
\end{table}

Do složených závorek následně můžete psát konkrétní změny. Například \texttt{background:~red;} nastaví pozadí prvku na červeno, \texttt{margin-right:~10px;} ho zase odsadí zprava o 10~pixelů apod. 
Styly můžete podobně jako u HTML vkládat do sebe – můžete si takto třeba pojistit, aby se změnil jen element s konkrétním rodičem. Můžete také přídávat pseudo třídy jako \texttt{:hover} (změní element, pokud na něj najede kurzor myši), \texttt{:first-child} (aplikuje se na prvního potomka rodiče) a mnohé další.

Důležité je také chápat rámečky a jak jdou po sobě. Jsou tři – vnitřní okraj (padding) určuje místo mezi obsahem a okrajem elementu (border), a vnější okraj (margin) zase oblast mezi borderem a ostatními prvky. U všech se dá nastavit velikost shora, zdola, zprava a zleva.

Ještě zmíním \texttt{display:~flex;}. Ten velmi zjednodušuje pozicování položek. Můžete je mít v řádku nebo sloupci, zarovnávat na začátek, střed, nebo konec, nastavovat mezi prvky mezery, a zároveň je i jednoduše roztahovat pomocí \texttt{flex-grow:~1;}.\cite{CSSFlexbox}

Ačkoliv už stránka vypadá mnohem lépe, zatím nic nedělá. Tlačítka Vás maximálně přesměrují na jinou stránku a pokud chcete vytvořit něco komplexnějšího než blog, je třeba pochopit ještě třetí důležitou technologii.