\section{HTML}

Na webu neexistuje důležitější technologie, než právě HTML. Bez tohoto značkovacího jazyku by na našem webu chyběl jakýkoliv obsah, slouží totiž k definování struktury stránky. Jeho syntax se skládá z párových (\texttt{<span>}potomek\texttt{</span>}) a nepárových (\texttt{<img/>}) značek, které se liší podle toho, zda v sobě obsahují potomky.

Zde je seznam několika základních elementů:

\begin{table}[h!]
\centering
\begin{tabular}{|>{\centering\arraybackslash}m{4cm}|>{\centering\arraybackslash}m{5cm}|}
\hline
\textbf{Element} & \textbf{Popis} \\ \hline
\texttt{span} & Jednoduchý text \\ \hline
\texttt{div} & Prázdný element \\ \hline
\texttt{p} & Odstavec \\ \hline
\texttt{button} & Tlačítko \\ \hline
\texttt{img} & Obrázek \\ \hline
\texttt{br} & Nový řádek \\ \hline
\texttt{h1, h2, h3, h4} & Nadpisy \\ \hline
\texttt{ol, ul} & Číslovaný/nečíslovaný seznam \\ \hline
\texttt{a} & Odkaz \\ \hline
\texttt{table} & Tabulka \\ \hline
\end{tabular}
\caption{Seznam základních elementů v HTML.}
\end{table}

Každý validní HTML dokument by měl mít dvě části. První je hlavička (head), ve které jsou definované informace, které na stránce nejsou vidět, jako její název, lokace skriptů a stylů, jazyk, font, nebo metadata. Druhá část je tělo (body). Tady už můžete používat všechny dříve zmíněné elementy a zobrazovat obsah samotné stránky.

Každý element má několik atributů, které můžete nastavit. \textit{Id} slouží jako unikátní identifikátor a \textit{class} zase reprezentuje skupinu prvků. Pomocí \textit{href} můžete prvku nastavit url adresu, a eventy jako \textit{onclick} nebo \textit{onmouseover} zase po splnění podmínky (kliknutí na element, přejetí myší) spustí definovaný skript.

V raných verzích se HTML dalo použít nejen k definování struktury stránky, ale i k jejímu stylování a pozicování jednotlivých elementů. Postupem času se ale toto paradigma stalo zbytečně komplexním, a tak se struktura a prezentace oddělily a vzniklo CSS.