\section{JavaScript}

Nástup JavaScriptu umožnil webovým vývojářům udržovat na stránkách aktuální data a přidat interaktivní prvky, zlepšující uživatelský zážitek. 

Ačkoliv se to tak nemusí podle názvu zdát, JavaScript má s Javou jen pramálo společného. Vytvořil ho v roce 1995 během deseti dnů zaměstnanec firmy Netscape, tehdy ještě pod jménem Mocha a LiveScript. Java se do jména dostala až kvůli popularitě stejnojmenného jazyka. Standardy pro něj vyvíjí organizace Ecma International, podle níž se značí i jeho verze (ES5, ES6, \ldots).\cite{JSIntroduction}\cite{JSOrigins}\cite{JSHistory}

Protože se jedná o jediný jazyk, který umí prohlížeče číst, a zároveň je jednoduchý na pochopení i programování, docela rychle se rozšířil. O to více, když v roce 2009 vzniklo Node.js a umožnilo ho používat i mimo webové prostředí. Od té doby v něm lze napsat opravdu téměř vše a i díky velkému množství dostupných knihoven (například v online repozitáři \href{https://www.npmjs.com/}{NPM}) se těší vysoké popularitě.

JavaScript má dva typy proměnných – let (dá se měnit) a const (neměnná). Do nich lze přiřazovat různé zabudované typy jako boolean, number (64bitové desetinné číslo), bigInt, string, array, date, objekt a několik dalších. Existuje i tři prázdné typy, jako je \texttt{undefined} (bez přiřazené hodnoty), \texttt{null} (prázdá hodnota) a \texttt{NaN} (Not-a-Number, doslova \textit{není číslo}). Typy se nicméně kontrolují až při běhu, takže se na špatné přiřazení často přichází, až když je příliš pozdě.\cite{JSBasics}\cite{JSTypes}

Pokud píšete podmínky, musíte dávat pozor na porovnávání. Dvě rovnítka porovnávají typy, tři zase hodnoty. Záměnou prvního rovnítka za vykřičník vznikne negace. $\&\&$ značí logický operátor AND a $||$ zase OR. Některé typy (undefined, null, NaN, nebo prázdné řetězce) se označují jako \textit{falsy} a v podmínkách reprezentují hodnotu \texttt{false}. Dále můžete používat ternary operátory ($x\ ?\ y : z$, pokud je $x$ pravda, vykoná se $y$, jinak $z$) nebo jeho zkrácenou verzi ($x\ \&\&\ y$, pokud je $x$ falsy, vykoná se $y$), které se převážně používá v Reactu pro podmíněné renderování.

Existuje i několik druhů cyklů. V obyčejném for cyklu nastavujete počáteční hodnotu, krok a podmínku, přičemž všechny argumenty jsou nepovinné. Nechybí ani podmíněný while a pro iteraci elementů v seznamu se nejlépe hodí forEach().

Jazyk umožňuje funkcionální i objektové programování. Třídy stejně jako u ostatních jazyků podporují konstruktory, dědičnost, zapouzdření, nebo interní metody (prefix \texttt{\#}). Objekty lze volně kombinovat s funkcemi. Ty mohou požadovat argumenty, které jsou buď kopií původních hodnot (primitivní typy), nebo referencí na původní proměnnou (objekty, pole, funkce). Každá funkce má návratový typ, který je v základu \texttt{undefined}.\cite{JSFunctions}\cite{JSFunctionalProgramming}