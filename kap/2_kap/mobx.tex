\section{MobX a state management}

To, že správa stavů v Reactu není úplně ideální, se ví už dlouho. Už jen to, že musíte data slučovat s vizuální částí, Vám téměř znemožňuje testování dat. A pokud by na jednom stavu záleželo více komponent, bylo by potřeba jej nějakým způsobem sdílet.

Existují desítky knihoven\cite{StateManagementLibs}, které Vám umožňují spravovat stav. Redux (první a nejpoužívanější), Hookstate (minimalistický, několik druhů stavů), Recoil (od Facebooku), nebo právě MobX, který jsem při tvorbě stránky použil. Ačkoliv se funkce každého z nich jmenují úplně jinak, jsou si dost podobné a mají jediný úkol. Uchovat stav mimo React.

MobX to dělá takto. V souboru poblíž kompozice si vytvoříte třídu a do ní vložíte všechny stavy, které ve Vaší kompozici potřebujete – aktuální text, barvu, obrázek, atd. A ke všem přidáte @observable dekorátor, který funguje podobně jako useState() hook. Pro aktivaci uvnitř Reactu ještě musíte všechny kompozice a komponenty, které MobX využívají, obalit observer() funkcí.

To ale není vše, co umí. Pokud potřebujete změnit několik observable proměnných najednou a předejít zbytečným překreslením, můžete jejich aktualizaci obalit do metody runInAction() a naznačit MobXu, aby vše vykonal najednou. MobX disponuje i funkcí podobnou již zmiňovanému hooku useMemo() - dekorátor @computed\cite{MobXComputed}, který se přepočítá pouze tehdy, pokud se některá ze závislostí (použitých observable proměnných) změní.