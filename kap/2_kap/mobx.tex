\section{MobX a state management}

Jednou z nevýhod Reactu je správa stavů. Pokud jich máte v komponentě několik, špatně se s nimi pracuje a data se nedají testovat skrz unit testy. Zároveň jediný způsob jak je sdílet, je dědit je z nadřazené komponenty.

Existují desítky knihoven\cite{StateManagementLibs}, které Vám umožňují efektivněji spravovat stav. Redux (první a nejpoužívanější), Hookstate (minimalistický, několik druhů stavů), Recoil (od Facebooku), nebo právě MobX, který jsem při tvorbě stránky použil. Ačkoliv se funkce každého z nich jmenují úplně jinak, jsou si dost podobné a mají jediný úkol. Uchovat stav mimo React.

MobX to dělá takto. Vytvoříte si třídu a do ní vložíte všechny stavy, které ve vaší komponentě potřebujete – aktuální text, barvu, obrázek, atd. A ke všem přidáte \texttt{@observable} dekorátor, který funguje podobně jako \texttt{useState()} hook. Pro aktivaci uvnitř Reactu ještě musíte všechny kompozice a komponenty, které MobX využívají, obalit \texttt{observer()} funkcí.

To ale není vše, co umí. Pokud potřebujete změnit několik observable proměnných najednou a předejít zbytečným překreslením, můžete jejich aktualizaci obalit do metody \texttt{runInAction()} a naznačit MobXu, aby vše vykonal najednou. MobX disponuje i funkcí podobnou již zmiňovanému hooku \texttt{useMemo()} - dekorátor \texttt{@computed}\cite{MobXComputed}, který se přepočítá pouze tehdy, pokud se některá ze závislostí (použitých observable proměnných) změní.