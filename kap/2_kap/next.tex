\section{Next.js}

Ať už je to směrování, optimalizace obrázků, nebo SEO, o většinu těchto věcí se musíte postarat sami. A často to není nic jednoduchého. Proto existuje framework Next.js\cite{NextJSDocs}, nástavba Node.js s podporou Reactu, který dokáže celý web posunout s minimem znalostí na úplně novou úroveň. Popsat ho celý by bylo na další dlouhodobou práci, proto tady zmíním alespoň základní výhody. Zároveň doporučuji projít si tuto příručku (\href{https://nextjs.org/learn}{https://nextjs.org/learn})\cite{NextJSLearn}, která je všechny vysvětluje do hloubky a ještě Vás naučí pár věcí navíc.

\subsection*{Routing}

Funkci, kterou využijete na každé stránce, je routing a tvorba cest. Už nemusíte endpointy udržovat v samostatných souborech – teď se budou tvořit podle adresářové struktury, neboli tak, jak je umístíte do složky jménem /app. Takže třeba /app/home/explore/page.tsx (každá stránka musí mít tento soubor, něco jako index.html) bude dostupná na url adrese www.example.com/home/explore.

Celý framework je napsaný tak, aby bylo SPA na prvním místě. SPA znamená Single Page Application a stojí na tom, že místo, aby se po změnění odkazu načetla znovu celá stránka, stačí jen stáhnout a změnit její část – uživatel tak nic nepozná a web se chová jako aplikace.

Ne všechny části jsou ale pokaždé potřeba znovu stahovat. Některé, třeba hlavička, budou po celém webu na stejném místě. Stejně jako má Next.js page.tsx soubory, tak má i layout.tsx, do kterého tyto neměnné komponenty a kompozice můžete zahrnout a ušetřit tak síťový výkon.

Pokud potřebujete jednoduše vytvořit API endpoint, můžete využít soubor route.tsx. Pro chyby je error.tsx, pro neplatné cesty not-found.tsx, a dokonce existuje i loading.tsx, který slouží jako placeholder a uživatel ho uvidí, pokud stránka musí čekat na nějaká data.

Pokud je část cesty dynamická, dá se umístit do složky, jejíž název je ohraničený hranatými závorkami. To se hodí například u uživatelských profilů. Jednoduché závorky zase slouží k logické organizaci, která se neprojeví v URL.

\subsection*{Serverové a klientské komponenty}

Pokud tvoříte komponentu, máte dvě možnosti, kde ji vykreslit\cite{NextJSS&CComponents} – na serveru\cite{NextJSServerComponents} nebo na klientovi\cite{NextJSClientComponents}. Obojí má své pro a proti. Na serveru můžete přímo volat do databáze, cachovat výsledky a použít je jinde, a posílat uživatelům již vygenerované stránky, takže nemusí čekat, než jim je React poskládá. Na druhou stranu zase nemůžete používat useState(), useEffect, poslouchat eventy, nebo přistupovat k DOMu a do prohlížeče. Mezi oběma způsoby je pomyslná síťová hranice, a je jen na Vás, kde si ji při vymýšlení architektury a stavby projektu stanovíte.

\subsection*{Optimalizace fontů a obrázků}

Pokud používáte na své stránce jakýkoliv font, co není v základu (například od Googlu), ukáže se stránka na klientovi nejdříve se základním fontem, a až poté, co se z dané adresy stáhne, se aplikuje ten správný. To může způsobit posun elementů, protože každý font má trochu jinou velikost a vzhled. A Google, ta samá firma, od které font nejspíš máte, vám pak může snížit celkové SEO hodnocení a zobrazovat Vaši stránku méně lidem. Next.js ale fonty stáhne již při kompilaci a pošle Vám je spolu s celou stránkou, takže je prohlížeč použije ihned.

Stejně jako fonty musíte mít optimalizované i obrázky. Na ty má Next.js vlastní <Image /> komponentu. Ta zabraňuje posunu elementů při načítání stránky, posílá velikosti obrázků úměrné k velikosti použitého displeje, šetří výkon tím, že obrázky načte až poté, co jsou vidět na displeji, a snaží se je všechny odesílat v moderních formátech, jako WebP nebo AVIF.

\subsection*{Streaming}

Next.js podporuje i streaming, který znáte ze všech dnešních sociálních sítí. Jde o postupné posílání obsahu stránky, například při jejím posunutí. K aplikování na stránky stačí obalit danou komponentu pomocí <Suspense />. Na podobném principu funguje i již zmiňovaný loading.tsx, který <Suspense /> aplikuje na celý soubor. K této komponentě se váže i tzv. Partial loading, který kombinuje serverové a klientské renderování a časem má potenciál stát se standardním způsobem pro tvorbu stránek v tomto frameworku.

\subsection*{Hooky}

Existuje několik hooků. Pro získání aktuální cesty se používá usePathname(). To můžete použít například u zvýraznění aktuálního odkazu v menu. UseSearchParams() zase podle klíče v URL najde jeho hodnotu (třeba id) a useRouter() umožní přistupovat k routeru a měnit URL nebo stránku znovu načíst.
