\section{Node.js}

Aby byl JavaScript v prohlížeči bezpečný, musí mít nějaká omezení. Například nemít možnost otevírat a číst vaše soubory v počítači, nebo se dostat mimo stránku do ostatních záložek a aplikací. Tím se jazyk stává omezeným pouze na tvorbu frontendu. To se ale změnilo s příchodem Node.js.\cite{NodeJS}

Node.js je open-source projekt, který má v sobě zabudovaný ten samý engine (V8), co prohlížeče jako Chrome, Edge, nebo Brave používají pro spouštění JavaScriptu. Kromě toho mu ale umožňuje manipulovat se systémem, takže v něm můžete bez problémů napsat celý webový server nebo cokoliv jiného, co lze dělat i s jinými vysokoúrovňovými jazyky.\cite{NodejsWiki}

Ačkoliv zde neexistují objekty pro manipulaci se stránkou nebo oknem (např. document nebo window), má několik vlastních modulů. Http pro tvorbu serveru, fs pro práci se soubory, nebo os s informacemi o systému. Díky svojí architektuře má i vlastní správu událostí. Node.js je zároveň důvod, proč vznikla databáze a následná správa knihoven NPM (Node Package Manager).