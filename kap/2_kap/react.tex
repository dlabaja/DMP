\section{JSX a React}

JSX je syntax, díky které se konečně můžete zbavit statických HTML souborů a psát UI kompletně v JavaScriptu, a to se všemi známými tagy, které jste používali doteď.\cite{JSX} Na stránce tak můžete jednoduše používat dynamické hodnoty a ani pro definici chování eventů si už nemusíte zakládat nové skripty.

Místo celých stránek nyní můžete tvořit samostatné komponenty, které jsou izolované, znovupoužitelné, a dají se ovládat zvenku. Ty následně můžete spojovat do kompozic reprezentujících části stránek a ty zase do stránek samotných. Celá struktura je tak mnohem modulárnější a chování komponent všude stejné.

K JSX je standardně potřeba knihovna React, která všechny elementy na stránce vykresluje. Každý z nich má svůj stav, a stejně jako u funkcí, i elementy by měly být deterministické a pro každou kombinaci vstupů mít maximálně jeden výstup. Stav se dá dále rozdělit na vnitřní a vnější podle jeho přístupnosti. Pokud se kterýkoliv z nich změní, React si toho všimne a element překreslí, tentokrát s novými daty.

Protože se s každým dalším překreslením znovu spustí celá funkce, která element definuje, nelze vnitřní stavy uchovávat v mutabilních ani globálních proměnných.\cite{ReactState} Místo toho React používá hooky, speciální funkce, které se dají během renderu volat. Pro uchování stavu mezi rendery se používá hook useState(). Při změně jeho obsahu se zavolá překreslení, ale hodnota se přenese. Pokud nic překreslovat nechcete, můžete hodnotu uložit do useRef()\cite{UseRef}.

Jak ale React vlastně funguje?\cite{ReactDeepDive} Pokaždé, když se změní stav, vytvoří virtuální kopii DOMu a v něm provede změny.\cite{VirtualDOM} Následně ho porovná s původním virtuálním DOMem, zjistí, co se změnilo, a následně se změny pokusí v jednom balíku aplikovat na reálný DOM. Aby při porovnávání ušetřil výkon, při detekci změny považuje za změněný element včetně všech jeho potomků.

Každý z elementů má určitou životnost. Ta začíná po připojení k DOMu (prvním vykreslení). Následně se kvůli změně stavů několikrát překreslí, než nadejde jeho čas a je ze stránky odpojen. Na to vše se dá reagovat useEffect() hookem.

Několik hooků má na starost zase optimalizaci výkonu. Pro zabránění zbytečnému přepočítávání hodnot se používá hook useMemo(), který vrací její naposled vypočítanou verzi. Podobné chování má i useCallback(), ten ale zase ukládá definici funkce.

To by mohlo jako takový rychlý úvod do Reactu stačit. Jak jste asi poznali, úplně mění způsob, jak vnímáme a tvoříme stránky. Teď je před námi ale nový problém. Data jsou smíchaná s UI. Ideální by bylo mít nějakou datovou třídu se všemi stavy, kterou by mohly používat některé větší kompozice. K tomuto účelu se používá některý z manažerů stavů.