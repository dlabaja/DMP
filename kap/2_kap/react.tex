\section{JSX a React}

JSX je syntax, která umožňuje zbavit se statických HTML souborů a začít psát UI kompletně v JavaScriptu, a to se všemi známými tagy, které jste používali doposud.\cite{JSX} Na stránce tak můžete jednoduše používat dynamické hodnoty a ani pro definici chování eventů si už nemusíte zakládat nové skripty.

Místo celých stránek nyní můžete tvořit samostatné komponenty, které jsou izolované a~znovupoužitelné. Ty následně můžete spojovat do kompozic reprezentujících části stránek a ty zase do stránek samotných. Celá struktura je tak mnohem modulárnější a chování komponent předvídatelnější.

K JSX je standardně potřeba knihovna React, která všechny elementy na stránce vykresluje. Každý z nich má svůj stav, a stejně jako u funkcí, i elementy by měly být deterministické a pro každou kombinaci vstupů mít maximálně jeden výstup. Stav se dá dále rozdělit na vnitřní a vnější podle jeho přístupnosti. Pokud se kterýkoliv z nich změní, React na to zareaguje a element překreslí, tentokrát s novými daty.

Protože se s každým dalším překreslením znovu spustí celá funkce, která element definuje, nelze vnitřní stavy uchovávat v mutabilních ani globálních proměnných.\cite{ReactState} Místo toho React používá hooky, speciální funkce, které se dají během renderu volat. Pro uchování stavu mezi rendery se používá hook \texttt{useState()}. Při změně jeho obsahu se zavolá překreslení, ale hodnota se přenese. Pokud nic překreslovat nechcete, můžete hodnotu uložit do \texttt{useRef()}\cite{UseRef}.

Jak ale React vlastně funguje?\cite{ReactDeepDive}. Pokaždé, když se změní stav, vytvoří virtuální kopii DOMu a v něm provede změny.\cite{VirtualDOM} Následně ho porovná s původním virtuálním DOMem, zjistí, co se změnilo, a následně se změny pokusí v jednom balíku aplikovat na reálný DOM. Aby při porovnávání ušetřil výkon, kromě změněného elementu překreslí i všechny jeho potomky. Nové elementy tvoří dynamicky pomocí JavaScriptu, proto vidíte jen prázdnou stránku, pokud ho zakážete. 

Každý z elementů má určitou životnost. Ta začíná po připojení k DOMu (prvním vykreslení). Následně se kvůli změně stavů několikrát překreslí, než nadejde jeho čas a je ze stránky odpojen. Na to vše se dá reagovat \texttt{useEffect()} hookem.

Několik hooků má na starost zase optimalizaci výkonu. Pro zabránění zbytečnému přepočítávání hodnot se používá hook \texttt{useMemo()}, který vrací její naposled vypočítanou verzi. Pro funkce se zase používá \texttt{useCallback()}.