\section{Sass, Less, CSS moduly a Tailwind}

Problémy s vývojem se nevyhnuly ani CSS. O jejich řešení se snaží různé preprocesory, konkrétně Sass\cite{Sass} (a jeho verze s CSS syntaxí SCSS) nebo Less\cite{Less}. Stejně jako TypeScript se jedná pouze o nástavbu, která se v tomto případě převádí zpět do CSS.

Jeden z problémů, které preprocesory řeší, je znovuvyužitelnost stylů. Nyní se dají organizovat do mixinů a ty následně používat jako jeden styl. Stejně jako funkce umožňují předávat parametry a s těmi dále pracovat. Například zapomocí nově přidaných podmínek a cyklů. Konzistenci webu napomáhají také proměnné, do kterých se nejčastěji ukládají barvy nebo velikosti písma a prvků a které se dají globálně používat v celém projektu. 

Ať už se preprocesor rozhodnete používat nebo ne, jistě vám život ulehčí CSS moduly.\cite{CssModules} Jak stránka roste, rostou i její styly, a může se jednoduše stát, že nová třída bude mít stejné jméno jako úplně jiná třída na druhé straně projektu. Pokud se to stane, styly se mohou navzájem ovlivňovat a velmi jednoduše vzhled stránky rozbít. Tento nástroj při skládání projektu všem třídám přidá do jména identifikátor, díky čemuž je každé jméno lokální.

Za zmínku ještě stojí Tailwind, framework, který všechny zmíněné technologie dokáže kompletně nahradit. Místo toho, aby vývojář tvořil třídy s vlastními styly, má již Tailwind třídy předdefinované a ty se následně kombinují mezi sebou. Nespornou výhodou je eliminace externích souborů se styly a rychlejší vývoj, výčet nevýhod zahrnuje horší čitelnost a nepřehlednost kódu kvůli dlouhým obsahům tříd. O tom, zda je lepší používat Tailwind nebo CSS se dodnes kvůli jejich rozdílnému pohledu na stylování stránek vedou spory. \cite{Tailwind}