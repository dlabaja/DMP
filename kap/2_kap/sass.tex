\section{Sass, Less a moduly}

O CSS už jste četli. A i tato technologie má překvapivě pár nevýhod. Jedna z největších je fakt, že nemůžete znovu použít už napsaný kód. Takže pokud chcete na více místech tučné bílé písmo o velikosti 12 pixelů, musíte ho vždy znovu definovat. A pokud si po nějakém čase uvědomíte, že by písmo mělo být větší, nezbývá Vám nic jiného, než jeho výšku na všech místech změnit a doufat, že jste na nic nezapomněl.

Tohle už se zřejmě několika lidem stalo, proto v roce 2006 vzniknul Sass\cite{Sass} a brzy po něm Less\cite{Less}. Podobně jako TypeScript se jednalo pouze o nadstavbu nad stávající jazyk, který se v tomto případě zkompiloval do CSS. Dokázal ale vyřešit několik do té doby neřešitelných problémů.

Proměnné usnadnily život nejednomu programátorovi. Pokud si designér na poslední chvíli rozmyslel velikost fontu, stačilo ho změnit jen na jednom místě. K tomu se vážou i mixiny – malé části kódu, na které se dá kdekoliv odkázat. Navíc přijímají i parametry, umožňující přizpůsobit chování mixinu zvenku. Kromě toho tyto nadstavby zvládají aplikovat jednotlivé styly v závislosti na podmínkách, umožňují používat cykly (většinou pro aplikování stylů na třídy a identifikátory s podobným názvem), a také dokáží zpracovat vnořování jednotlivých stylů, což dělá soubory menší a kód čitelnější.

Jako poslední zmíním CSS moduly.\cite{CssModules} Jak stránka roste, rostou i její styly, a může se jednoduše stát, že nová třída bude mít stejné jméno jako úplně jiná třída na druhé straně projektu. Pokud se to stane, styly se mohou navzájem ovlivňovat a velmi jednoduše vzhled Vaší stránky rozbít. Tato jednoduchá knihovna dělá z globálních stylů lokální a při kompilaci ke každé třídě přidá vygenerovaný identifikátor. Protože jsou od teď ale názvy pravidel dynamické, musíte začít psát HTML pomocí JavaScriptu nebo TypeScriptu, který je umí zadat za Vás. Ideálně potřebujete najít způsob, jak HTML udělat více abstraktní a celý proces vykreslování zajišťovat skriptovacím jazykem.