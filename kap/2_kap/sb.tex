\section{Storybook a mocky}

Doteď jste mohli své komponenty testovat a vidět jen na finální stránce. To je ale poněkud nepraktické, protože musíte čekat na kompilaci celého projektu, používáte reálná data a komponenty nejsou izolované. Pro to se používá knihovna Storybook. V ní si můžete ve webovém rozhraní všechny komponenty prohlédnout a jednoduše i bez znalosti kódu měnit jejich stav. Tím usnadníte práci i designerům, produktovým manažerům a testerům.

Jediné, co potřebujete, je (kromě instalace) vytvořit pro každou komponentu soubor s příponou \textit{*.stories.*}. V něm definujete stavy komponenty, které půjdou měnit, a ke kterým Storybook automaticky přiřadí odpovídající vizuální prvek (například řetězce můžete psát to textového pole, seznamy vybírat z dropdownu a rozsah vybírat posuvníkem) a následně mu řeknete, kterou komponentu má zobrazit.

Pro využití Storybooku naplno je ještě třeba přerušit komunikaci se serverem. Každý projekt na to jde po svém. Pokud například tlačítko volá API skrze onClick() metodu, pak ji stačí jednoduše zpřístupnit zvenku a nechat ji prázdnou. Ne vždy je to ale takto jednoduché a kód komponent by se kvůli Storybooku nikdy neměl modifikovat. Lepší způsob je tedy vytvořit mocky\cite{Mocks}. Stačí vytvořit úplně stejnou třídu, jako ta, kterou chcete „mocknout“, jen s falešnými metodami, které místo volání na server vrací falešná data. Ní pak nahradíte reálnou implementaci třídy. Kromě Storybooku můžete mocky používat i v unit testech, ke kterým se ještě dostanu.