\subsection{MobX store a validace dat}
Jak funguje MobX jsem popisoval ve druhé části této práce. Používáme ho hlavně z toho důvodu, abychom měli data izolovaná od Reactu, ale zároveň je v něm mohli jednoduše používat. Můžeme je tak ukládat do tříd a ty následně bez problémů testovat.

Pro odlišení od komponent má každý takový soubor příponu \textit{*.store.*}. Třída v něm je složená z polí obsahujících všechna data, která se budou na stránce používat, jejich náležitých getterů (a většinou i setterů) a konstruktoru. Aby na změny dat reagoval React, musí být všechna pole označena dekorátorem @observable.

Konstruktor by měl správně jen dosadit základní hodnoty, spustit nějaké interní funkce a zavolat \texttt{makeObservable()} funkci, čímž aktivuje MobX.
Pokud chci získat nějaká externí data, musím to udělat zvlášť. Proto má každá ze stránek svoji veřejně přístupnou metodu \texttt{init()}, kterou v Reactu volám po prvním renderu a narozdíl od konstruktoru může být asynchronní. Tím ovšem vzniká další problém – když se stránka načte, chybí jí data. Z toho důvodu jsem si vytvořil vlastní funkci \texttt{loadDataAsync()}, kterou všechny stránky používají, spolu s \texttt{isInitialized} stavem. Ten se změní, jakmile se všechna data načtou, a díky tomu vím, kdy stránku zobrazit uživateli, a kdy ji nahradit načítací animací.

Každá z mých tříd má dvě speciální pole. \texttt{FormChanged} se nastaví na \texttt{true} po tom, co změníte jakákoliv data na stránce – třeba vlastní přezdívku. Tím se Vám zároveň odemkne tlačítko s odesláním dat, čímž jsem alespoň částečně zabránil tomu, aby náš server zaplňovaly prázdné požadavky. Jakmile data odešlete, \texttt{formChanged} se opět nastaví na\texttt{ false} a tlačítko zamkne.

\texttt{FormLoading} se zase nastaví na \texttt{true} po tom, co data uložíte. Zamkne všechna pole, aby hodnoty zůstaly během odesílání konzistentní, a zůstane tak, dokud server nepošle odpověď.

Metoda \texttt{init()} není jediná, která potřebuje přistupovat k nějaké externí API. Pro odeslání dat na server má každá ze tříd i metodu \texttt{save()}, některé z nich pak mají metody například na propojení Facebooku nebo Instagramu, nebo mazání a nahrávání fotky na profilu.

U nastavení účtu a změny hesla jsem se nevyhnul ani validaci dat. Pokaždé, když kliknete na tlačítko pro uložení, spustí se metoda \texttt{validate()}. Ta aktualizuje hodnoty jako \texttt{nicknameIsValid}, nebo \texttt{emailIsValid} podle toho, zda nejsou prázdné nebo v neplatném formátu. Getterem \texttt{isValid()} se dá následně zjistit, zda validace celého formuláře prošla.

Pokud bylo vráceno \texttt{true}, spustí se část, kterou mají všechny stránky stejnou – funkce \texttt{saveDataAsync()}. Té předám \texttt{save()} metodu dané stránky, a také již zmíněný \texttt{formLoading} a \texttt{formChanged}, se kterými podle odpovědi serveru manipuluje. Ať už je odpověď jakákoliv, vždy se dole zobrazí Toast, díky němuž uživatel zjistí, zda se operace zdařila, server neodpovídá, nebo se například neshodují hesla.