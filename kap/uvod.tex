\chapter*{Úvod}\addcontentsline{toc}{chapter}{Úvod}\markboth{Úvod}{}

Asi spousta z vás už někdy vytvořila webovou stránku. Někteří z vás k její tvorbě dokonce využili i jeden z frameworků. Jak hluboko ale tato králičí nora jménem tvorba frontendu vlastně může jít?

Podobnou otázku jsem si pokládal koncem května 2024 po vstupu do startupu Worldee. Z webu jsem uměl jen HTML, CSS a základy JavaScriptu a zhruba tušil, k čemu je Sass. Během léta jsem si nejen tyto znalosti upevnil, ale navíc se naučil i pro mě úplně nové technologie, jako TypeScript, React, MobX, Next.js, nebo práci se Storybookem. Nakonec jsem v těchto technologiích vytvořil stránku s uživatelským nastavením, která bude na webu Worldee ještě několik dalších let. A přesně o tom bude tato práce.

V první kapitole se pokusím objasnit, co je prací a posláním společnosti. Pojem \uv{cestovka 2.0} nebo \uv{cestovatelská platforma} je totiž sice docela jednoznačný, ale zdaleka nepokrývá vše, co firma dělá. V další kapitole se pokusím popsat jednotlivé technologie, od těch základních, které většina lidí zná, jako HTML, CSS a JavaScript, přes
ty pokročilejší, jako React, TypeScript, nebo Next.js, až po některé nástroje používané na denní bázi, například next-intl a Jest. A nakonec se pokusím tvorbu takové stránky objasnit. S jakými komponentami jsem pracoval, jaké designy jsem skládal a jak vypadá jejich datová struktura. Řada přijde i na zmíněnou ukázku architektury webu.

Ale nyní – co je to to Worldee, jak vzniklo, a proč má potenciál stát se příštím \uv{jednorožcem}?