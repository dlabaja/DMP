\chapter*{Závěr}\addcontentsline{toc}{chapter}{Závěr}\markboth{Závěr}{}

Jak jste asi zjistili, svět webu je komplikovanější, než se může na první (nebo i druhý) pohled zdát. Už dávno mezi sebou neposíláme statické HTML dokumenty a s příchodem stylů, JavaScriptu, JQuery, a prvních frontendových frameworků se toho hodně změnilo. Nové, někdy opravdu revoluční knihovny vycházejí téměř na denní bází a je stále těžší držet krok. Dokonce i popsané technologie jsou jen špička ledovce a je dost možné, že za pár let už některé z nich ani nebudou existovat.

A ačkoliv jsou nové technologie komplikované, díky jejich abstrakci je psaní a hlavně následné škálování webu mnohem jednodušší než dřív. Místo šablon používáme komponenty a kompozice, místo stylů CSS moduly a Tailwind, JavaScript je typově bezpečný, zkompilovaný kód je díky bundlerům malý a efektivní, před chybami nás chrání testy, lokalizaci můžou překladatelé dělat přes webové rozhraní, a spoustu dalších větších a menších vylepšení pro uživatele i vývojáře.

Po dopsání tohoto projektu jsem se účastním vývoje i jiných částí webu. Opravil a vyčistil jsem Text editor, vytvořil komponentu s telefonním číslem a výběrem předvolby, přidal vyhledávání do mobilního Combo boxu, nebo mapuju data ze scraperu na naše API. Tvorba webů je něco, v čem jsem se momentálně našel, a co ještě nějakou chvíli budu dělat. A chtěl bych poděkovat celému vývojovému týmu, že mi pomáhají na cestě za jeho pochopením.